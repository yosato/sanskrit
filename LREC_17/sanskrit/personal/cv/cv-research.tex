% LaTeX Curriculum Vitae Template
%
% Copyright (C) 2004-2009 Jason Blevins <jrblevin@sdf.lonestar.org>
% http://jblevins.org/projects/cv-template/
%
% You may use use this document as a template to create your own CV
% and you may redistribute the source code freely. No attribution is
% required in any resulting documents. I do ask that you please leave
% this notice and the above URL in the source code if you choose to
% redistribute this file.

%% MACROS ARE FOUND IN cv-common.tex.

\documentclass[a4paper]{article}

\usepackage{cv}

\def\name{Yo Sato}
\def\version{research}
\def\website{{\tt\small http://satoama.co.uk/yo}}

%THIS IS THE MACRO FILE WHERE YOU FIND CORE INFO

%%% Local Variables: 
%%% mode: latex
%%% TeX-master: t
%%% End: 
\def\personal{%
\section*{Personal}

\begin{itemize}
\item Permanent resident in the United Kingdom. Japanese national.
\end{itemize}}

\def\education{%
\section*{Education}

\begin{itemize}
  \item Ph.D. Computational Linguistics, King's College London, U.K., 2008.

  \item M.Sc. Computational Linguistics (with Distinction), King's College London, U.K., 2003.

  \item Diploma in Computing, Open University, U.K., 2000.

  \item M.A. Applied Linguistics, Birkbeck College London, U.K., 1998.


  \item B.A. Philosophy, Waseda University, Japan, 1990.
\end{itemize}}


\def\languages{\section*{Language skills}
\begin{itemize}
\item Japanese (native): Lived in Japan for 25 years from birth. The
  language of daily use at home.

\vspace{-1mm}

\item  English (fluent): Have been living in London for 20 years. The language of
  daily use at work. Also the language in which I completed my higher
  degrees and conduct academic research.

\vspace{-1mm}

\item  French (advanced): Language of daily use for 4 years while living in France. Certified B2.

\vspace{-1mm}

\item  Korean (intermediate): Have been learning for five years. Can read documents
  and conduct basic conversations.
\end{itemize}

}



\def\computer{\section*{Computer skills}

Programming languages: Python, Prolog, Perl, C$^{++}$\\
OS: Unix (Linux) system operations and shell scripting\\
Text processing: sed, awk, Perl and Python\\
Other tools: \LaTeX, Git, NLTK, Weka, RNNLM

\def\researchWeb{{\tt\small http://homepages.stca.herts.ac.uk/\textasciitilde{}ys08aah}}
\def\papersWeb{\researchWeb{}{\tt\small /papers.html}}



%\def\publications_j{\section*{主要論文}

%\noindent{\smaller (全論文リストは\papersWeb{}を参照ください。またPDFのダウンロードが可能です。)}
%\bigskip
%{\fontspec{DejaVu Serif}
%\papers
%}
%}



\begin{document}

% Place name at left
{\huge \name}

\smallskip

Ph.D. and M.Sc. in Computational Linguistics, M.A. in Applied Linguistics\\

%\noindent Director, Satoama Language Services Ltd.

\vspace{0.25in}

\begin{minipage}{0.4\linewidth}

 21 Coombe Gardens\\
  New Malden, United Kingdom\\
KT3 4AB\\

\end{minipage}
\begin{minipage}{0.43\linewidth}
  \begin{tabular}{ll}
    Phone/Fax: & (+44) (0)20-82881419 \\
    Email: & {\tt satoama@gmail.com} \\
    Website: & \website{} 

  \end{tabular}
\end{minipage}

\personal

\section*{Research interests and skills}

\begin{itemize}
\item  Language model building (statistical: N-gram including skip and clustered, neural net and word embeddings, symbolic: finite state techniques)\\
  Machine learning (various including: kNN and neural net for classification, k-means and hierarchical technique for clustering, sequential labelling with HMM)

  Computational modelling of language acquisition
  (phonology, semantics and syntax)\\
  Natural language parsing and theoretical syntax and semantics (HPSG,
  Dynamic Syntax, Montague Grammar)

  Usage of python-based tools (NLTK, sklearn, RNNLM)\\
  Parser building with Prolog\\
  Text processing (with Python, awk and sed)\\
  Phonetic/phonological analysis with Praat and HTK

\end{itemize}


\education

\section*{Research positions}

\begin{itemize}

\item \hspace{-2mm}Natural Language Processing Research Engineer, MyScript Lab,  2012--to date.

\hspace{-2mm}Duties: Building principally statistical language models for handwriting recognition and input comple- tion and prediction. Main responsibilities include: morphological processing and learning algorighms based on corpora, as applicable to a wide range of languages (but with a special focus on Japanese and Korean); various N-Gram modelling with multi-word expression detection capability; adaptation of models to individuals and dialects

\item \hspace{-.2cm}Research Fellow in Computational Linguistics,
  University of Hertfordshire, Jan 2009--Mar 2015 (Visiting Fellow from Mar 2012)

\hspace{-.2cm}Duties: Building a language acquisition model, including implementations of  grammar induction and phonological word discovery algorithms for `ITALK' (Integration and Transfer of Action and Language Knowledge in Robots), a European FP7 Framework Project  \\

\item \hspace{-.2cm}Research Assistant, Computer Science Dept, Queen
  Mary University of London, Jan 2008-- Jan 2009.

\hspace{-.2cm}Duties: A year contract for `Dynamics of Conversational
Dialogue', an Economic and Social Science Research Council (UK)
project awarded to King's and Queen Mary Colleges, London University. Responsible for building a parser based on the Dynamic Syntax formalism and a formal model of ellipsis.

\item \hspace{-.2cm}Doctoral Trainee, Computer Science Dept, King's
  College London, Sep 2004-- Feb 2008.

\hspace{-.2cm}Duties: Construction of natural language parser based on
HPSG for freer word order languages, supervised by Professor Shalom
Lappin, funded by the Engineering and Physical Science Research Council, U.K. 
\end{itemize}



\section*{Teaching}
\begin{itemize}

\item Lecturer/tutor, Imperial College London,  2003--2012\\
   M.Sc. in Scientific, Technical and Medical Translation \\
   Language Technology Module and English-Japanese Practical Translation Module

\item Teaching Assistant, King's College London. 2006.\\
    B.Sc. module `Natural Language Processing in Prolog' by Professor S. Lappin.\\
   Tutorials (hands-on programming sessions), exercise setting and assignment marking



\item Tutor, Bath University, 2006--2008, U.K.\\
   M.A. in Translation Studies, \\Supervision on dissertations.


\end{itemize}

\section*{Other employment and qualifications}

\begin{itemize}

\item Japanese Service, BBC World, 1996--2002

Duties: Translation and coordination related to the BBC news service


\item Qualified Member, Institute of Translation and Interpreting, U.K. (MITI, since 2002)
\end{itemize}


%%===========================================================================%%

\section*{Selected publications}

\noindent{\smaller (Full list is available on \website{}/papers.html)}

\bigskip

\include{publications}

\papers



\section*{Conference / workshop organisation}
\begin{itemize}
\item {\em Semantics and Pragmatics of Dialogue (SemDial) 2008, King's College London}, Publication chair
\item  {\em Special session on phonological word learning, International
    Conference on Development and Learning (ICDL) 2011, Frankfurt,
    Germany}, Session organiser

\end{itemize}


\languages


\computer



%%---------------------------------------------------------------------------%%

\section*{Interests}
Cricket, Cooking, Classical music


% Footer
\begin{center}
  \begin{footnotesize}
    Last updated: \today\\A more updated version may be available at: \\
    \href{\footerlink}{\texttt{\website{}/cv-research.pdf}}
  \end{footnotesize}
\end{center}

\end{document}
