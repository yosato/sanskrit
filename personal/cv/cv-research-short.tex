% LaTeX Curriculum Vitae Template
%
% Copyright (C) 2004-2009 Jason Blevins <jrblevin@sdf.lonestar.org>
% http://jblevins.org/projects/cv-template/
%
% You may use use this document as a template to create your own CV
% and you may redistribute the source code freely. No attribution is
% required in any resulting documents. I do ask that you please leave
% this notice and the above URL in the source code if you choose to
% redistribute this file.

\documentclass[a4paper]{article}

\usepackage{hyperref}
\usepackage{geometry}

% Comment the following lines to use the default Computer Modern font
% instead of the Palatino font provided by the mathpazo package.
% Remove the 'osf' bit if you don't like the old style figures.
\usepackage[T1]{fontenc}
\usepackage[sc,osf]{mathpazo}

% Set your name here
\def\name{Yo Sato}

% Replace this with a link to your CV if you like, or set it empty
% (as in \def\footerlink{}) to remove the link in the footer:
\def\footerlink{http://homepages.stca.herts.ac.uk/\textasciitilde{}ys08aah/cv-research-short.pdf}

% The following metadata will show up in the PDF properties
\hypersetup{
  colorlinks = true,
  urlcolor = black,
  pdfauthor = {\name},
  pdfkeywords = {economics, statistics, mathematics},
  pdftitle = {\name: Curriculum Vitae},
  pdfsubject = {Curriculum Vitae},
  pdfpagemode = UseNone
}

\geometry{
  body={6.5in, 8.5in},
  left=1.0in,
  top=1.25in
}

% Customize page headers
\pagestyle{myheadings}
\markright{\name}
\pagestyle{empty}


% Custom section fonts
\usepackage{sectsty}
\sectionfont{\rmfamily\mdseries\Large}
\subsectionfont{\rmfamily\mdseries\itshape\large}

% Other possible font commands include:
% \ttfamily for teletype,
% \sffamily for sans serif,
% \bfseries for bold,
% \scshape for small caps,
% \normalsize, \large, \Large, \LARGE sizes.

% Don't indent paragraphs.
\setlength\parindent{0em}

% Make lists without bullets
\renewenvironment{itemize}{
  \begin{list}{}{
    \setlength{\leftmargin}{1.5em}
%    \setlength{\parindent}{-2cm}
  }
}{
  \end{list}
}

\addtolength{\topmargin}{-.5cm}
\addtolength{\textheight}{2.2cm}

\begin{document}

%\vspace{-1cm}

% Place name at left
{\huge \name}

\smallskip

Ph.D. and M.Sc. in Computational Linguistics, M.A. in Applied Linguistics

% Alternatively, print name centered and bold:
%\centerline{\huge \bf \name}

\vspace{0.25in}

\begin{minipage}{0.5\linewidth}
  Research Fellow in Computational Linguistics\\
  Rm 114, Science and Technology Research Institute \\
  University of Hertfordshire \\
  College Lane, HA10 9AB
\end{minipage}
\begin{minipage}{0.45\linewidth}
  \begin{tabular}{ll}
    Phone: & (+44) (0)1707-28-4266 \\
    Fax: &  (+44) (0)20-8288-1419 \\
    Email: & {\tt satoama@gmail.com} \\
    Website: & {\tt\small http://homepages.stca.herts.ac.uk/\textasciitilde{}ys08aah} \\
  \end{tabular}
\end{minipage}

\section*{Personal}

\begin{itemize}
%\item Born on September 29, 1895.
\item Permanent resident in the United Kingdom. Japanese national.
\end{itemize}

\section*{Key research interests and research-related skills}

\begin{itemize}
\item Computational modelling of language acquisition
  (phonology, semantics and syntax)\\ Grammar induction algorithms\\ Natural
  language parsing\\ Syntax, semantics and their interface (HPSG,
  Dynamic Syntax, Montague Grammar)

  Constraint-based grammar and parser building (with Prolog and Python)\\
  Text processing (with Python, awk and sed)\\
  Speech processing: acoustic model creation, recognition and forced alignment with HTK\\
  Phonetics/phonology: phonemic transctiption, speech (including  prosodic) analysis with Praat\\
  Statistics: inference (Bayesian or Frequentist), significance tests and logistic regression with R\\
  Machine learning (kNN with TiMBL, Bayesian or other classification/clustering with WEKA)\\


\end{itemize}

\vspace{-.6cm}

\section*{Education}

\begin{itemize}
  \item Ph.D. Computational Linguistics, King's College London, U.K., 2008.

  \item M.Sc. Computational Linguistics (with Distinction), King's College London, U.K., 2004.

  \item M.A. Applied Linguistics, Birkbeck College London, U.K., 1999.

  \item B.A. Philosophy, Waseda University, Japan, 1989.
\end{itemize}


\section*{Research positions}

\begin{itemize}
\item \hspace{-.2cm}Research Fellow in Computational Linguistics,
  University of Hertfordshire, 2009--to date. 

\hspace{-.2cm}Duties: Building a language acquisition model, including implementations of  grammar induction and phonological word discovery algorithms for `ITALK' (Integration and Transfer of Action and Language Knowledge in Robots), a European FP7 Framework Project  \\

\item \hspace{-.2cm}Research Assistant, Computer Science Dept, Queen
  Mary University of London, 2008--2009.

\hspace{-.2cm}Duties: A year contract for `Dynamics of Conversational
Dialogue', an Economic and Social Science Research Council (UK)
project awarded to King's and Queen Mary Colleges, London University. Responsible for building a parser based on the Dynamic Syntax formalism and a formal model of ellipsis.

\vspace{.4cm}

\item \hspace{-.2cm}Doctoral Trainee, Computer Science Dept, King's
  College London, 2004--2008.

\hspace{-.2cm}Duties: Construction of natural language parser based on
HPSG for freer word order languages, supervised by Professor Shalom
Lappin, funded by the Engineering and Physical Science Research Council, U.K. 
\end{itemize}



\section*{Teaching}
\begin{itemize}

\item Lecturer/tutor, Imperial College London,  2003--to date\\
   M.Sc. in Scientific, Technical and Medical Translation \\
   Language Technology Module and English-Japanese Practical Translation Module

\item Teaching Assistant, King's College London. 2005--2006.\\
    B.Sc. module `Natural Language Processing in Prolog' by Professor S. Lappin.\\
   Tutorials, lectures and exam setting/marking



\item Tutor, Bath University, 2006--2008, U.K.\\
   M.A. in Translation Studies, \\Supervision on dissertations.


\end{itemize}
%%===========================================================================%%


\section*{Selected publications}

{\em Underspecified types and the semantic bootstrapping of common nouns and adjectives: a simulation with a robot's sensory data} (with Tam Wailok), the Logic and Engineering of Natural Language Semantics (LENLS8), Takamatsu, Japan, 2011.\smallskip\newline
{\em Towards Using Prosody to Scaffold Lexical Meaning in Robots} (co-authored), the International Conference on Development and Learning (ICDL), Frankfurt, Germany, 2011.\smallskip\newline
`Local Ambiguity, Search Strategies and Parsing in Dynamic Syntax' in
Kempson et al (eds) {\em The Dynamics of Lexical Interfaces}, CSLI, 2011.\smallskip\newline
%{\em How much raw sounds can do for word learning: learning by `repeating' in human-robot interaction} (co-authored), 2011, Frontiers in .\smallskip\newline
{\em An Integrated Three-Stage Model towards Grammar Acquisition} (co-authored), the International Conference on Development and Learning (ICDL), Michigan, U.S., 2010\smallskip\newline
{\em What is Needed for a Robot to Acquire Grammar? Some Underlying Primitive Mechanisms for the Synthesis of Linguistic Ability} (co-authored), IEEE Transactions on Autonomous Mental Development, 1(3), 2009\smallskip\newline
{\em Grammar Resources for Modelling Dialogue Dynamically} (co-authored), Cognitive Neurodynamics 3(4), 2009\smallskip\newline
{\em Towards a Unified Account of Adjuncts} (with Tam Wailok), HPSG 2008, Kyoto, Japan.\smallskip\newline
%{\em An Analysis of Pseudopartitives and Mearure Phrases that Says No to Extra Rules} (co-authored), HPSG 2008, Kyoto, Japan.\smallskip\newline
{\em Lexicalised Parsing of German V2}, German Parsing Workshop, Association of Computational Linguistics, Columbus, Ohio, U.S., 2008\smallskip\newline
{\em A Proposed Lexicalised Linearisation Grammar: a monostratal alternative}, HPSG 2006, Varna, Bulgaria.\smallskip\newline
%{\em Lexicalising Word Order Constraints for Implemented Linearisation Grammar}, Conference of European Association of Computational Linguistics, 2006, Trento, Italy.\smallskip\newline
%{\em Constrained Free Word Order Parsing}, Computational Linguistics U.K., 2006, Milton Keynes, U.K.

\vspace{-.6cm}

\section*{Conference / workshop organisation}
\begin{itemize}
\item {\em Semantics and Pragmatics of Dialogue (SemDial) 2008, King's College London}, Publication chair
\item  {\em Special session on phonological word learning, International
    Conference on Development and Learning (ICDL) 2011, Frankfurt,
    Germany}, Session organiser

\end{itemize}

\vspace{-.5cm}

%\section*{Non-academic employment and qualifications}

%\begin{itemize}
%\item Japanese Service, BBC World, 1996--2002

%Duties: Translation and coordination related to the BBC news service


%\item Qualified Member, Institute of Translation and Interpreting, U.K. (MITI, since 2002)
%\end{itemize}

\section*{Languages (natural ones!)}

Japanese (native), English (fluent), German (advanced), Korean (intermediate), British Sign Language (Intermediate)

\vspace{-.3cm}

\section*{Computer skills}

OS: Unix (Linux) system operations and shell scripting\\
Languages: Python, Prolog, C and C$^{++}$, Haskell\\
Web: HTML, Javascript, Apache administration\\
Tools: \LaTeX, Subversion, XML, gettext localisation


%%---------------------------------------------------------------------------%%

%\section*{Interests}



% Footer
\begin{center}
  \begin{footnotesize}
    Last updated: \today\\A more updated version may be available at: \\
    \href{\footerlink}{\texttt{\footerlink}}
  \end{footnotesize}
\end{center}

\end{document}
