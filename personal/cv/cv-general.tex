\documentclass[a4paper]{article}

\usepackage{cv}

\addtolength{\topmargin}{-.7cm}
\addtolength{\textheight}{4cm}
\addtolength{\oddsidemargin}{-.2cm}
\addtolength{\evensidemargin}{-.2cm}
%\addtolength{}{}

\def\name{Yo Sato}
\def\version{general}


%%% Local Variables: 
%%% mode: latex
%%% TeX-master: t
%%% End: 
\def\personal{%
\section*{Personal}

\begin{itemize}
\item Permanent resident in the United Kingdom. Japanese national.
\end{itemize}}

\def\education{%
\section*{Education}

\begin{itemize}
  \item Ph.D. Computational Linguistics, King's College London, U.K., 2008.

  \item M.Sc. Computational Linguistics (with Distinction), King's College London, U.K., 2003.

  \item Diploma in Computing, Open University, U.K., 2000.

  \item M.A. Applied Linguistics, Birkbeck College London, U.K., 1998.


  \item B.A. Philosophy, Waseda University, Japan, 1990.
\end{itemize}}


\def\languages{\section*{Language skills}
\begin{itemize}
\item Japanese (native): Lived in Japan for 25 years from birth. The
  language of daily use at home.

\vspace{-1mm}

\item  English (fluent): Have been living in London for 20 years. The language of
  daily use at work. Also the language in which I completed my higher
  degrees and conduct academic research.

\vspace{-1mm}

\item  French (advanced): Language of daily use for 4 years while living in France. Certified B2.

\vspace{-1mm}

\item  Korean (intermediate): Have been learning for five years. Can read documents
  and conduct basic conversations.
\end{itemize}

}



\def\computer{\section*{Computer skills}

Programming languages: Python, Prolog, Perl, C$^{++}$\\
OS: Unix (Linux) system operations and shell scripting\\
Text processing: sed, awk, Perl and Python\\
Other tools: \LaTeX, Git, NLTK, Weka, RNNLM

\def\researchWeb{{\tt\small http://homepages.stca.herts.ac.uk/\textasciitilde{}ys08aah}}
\def\papersWeb{\researchWeb{}{\tt\small /papers.html}}



%\def\publications_j{\section*{主要論文}

%\noindent{\smaller (全論文リストは\papersWeb{}を参照ください。またPDFのダウンロードが可能です。)}
%\bigskip
%{\fontspec{DejaVu Serif}
%\papers
%}
%}



\include{publications}

\begin{document}

% Place name at left
{\huge \name}

\smallskip

Ph.D. and M.Sc. in Computational Linguistics, M.A. in Applied Linguistics\\
Qualified Translator (English $\rightarrow$ Japanese)\\
Permanent resident in the United Kingdom. Japanese national.

\bigskip

\begin{minipage}{0.3\linewidth}
  21 Coombe Gardens\\
  New Malden\\
  KT3 4AB
\end{minipage}
\begin{minipage}{0.45\linewidth}
  \begin{tabular}{ll}
    Phone/Fax: &  (+44) (0)20-8288-1419 \\
    Email: & {\tt satoama@gmail.com} \\
    Websites:     & {\tt\small http://www.satoama.co.uk/yo}
  \end{tabular}
\end{minipage}


%\personal{}

%\vspace{-6mm}

\education{}

\vspace{-5mm}

\section*{Employment}

\begin{itemize}

\item \hspace{-2mm}Natural Language Processing Research Engineer, Nuance Communications U.K., Nov 2016--to date.

\hspace{-2mm}Duties: Building natural language understanding models principally for Japanese for the use of speech recognition engine. 


\item \hspace{-2mm}Natural Language Processing Research Engineer, MyScript Lab,  Aug 2012--Sep 2016.

\hspace{-2mm}Duties: Building principally statistical language models for handwriting recognition and input comple- tion and prediction. Main responsibilities include: morphological processing and learning algorighms based on corpora, as applicable to a wide range of languages (but with a special focus on Japanese and Korean); various N-Gram modelling with multi-word expression detection capability; adaptation of models to individuals and dialects

\item \hspace{-.2cm}Research Fellow in Computational Linguistics,
  University of Hertfordshire, Jan 2009--Mar 2012. 

\hspace{-.2cm}Duties: Building a language acquisition model, in particular grammar induction and phonological word discovery algorithms, for `ITALK' (Integration and Transfer of Action and Language Knowledge in Robots), a European FP7 Framework Project

\item \hspace{-.2cm}Research Assistant, Computer Science Dept, Queen
  Mary University of London, Jan 2008--Jan 2009.

\hspace{-.2cm}Duties: Implementing a parser that handles constructions typical in dialogue, such as ellipsis, based on the Dynamic Syntax formalism, for `Dynamics of Conversational
Dialogue', an Economic and Social Science Research Council (UK)
project. 

\item\hspace{-.2cm}Japanese Service, BBC World Television, 1996--2002

Duties: Translation and coordination related to the BBC news and documentary services, including simultaneous interpreting of news and subtitling and voice-over of documentary programmes  

\end{itemize}

\section*{Qualifications and skills}

\subsection*{Language skills}
\begin{itemize}
\item Japanese (native): Lived in Japan for 25 years from birth. The
  language of daily use at home.

\vspace{-1mm}

\item  English (fluent): Living in England for 20 years. The language of
  daily use at work. Also the language in which I completed my higher
  degrees and conduct academic research.

\vspace{-1mm}

\item  French (advanced): Language of daily use for 4 years while living in France. Certified B2.

\vspace{-1mm}

\item  Korean (intermediate): Learning for five years. Can read documents
  and conduct basic conversations.
\end{itemize}

\subsection*{General computer skills}

\begin{itemize}
 \item OS: Unix (Linux) system operations and shell scripting\\
  Programming languages: Python, Prolog, C$^{++}$, Java\\
  Web: HTML5 and Javascript, Apache server administration
   Other tools: SQL, \LaTeX, Git, XML

\end{itemize}

\subsection*{Language processing skills}

\begin{itemize}
\item Constraint-based grammar and parser building (with Prolog and Python)\\
  Text processing with Python, awk and sed\\
  Speech processing: acoustic model creation, recognition and forced alignment with HTK\\
  % Phonetics/phonology: phonemic transctiption, speech (including  prosodic) analysis with Praat\\
  Statistical language learning: probabilistic grammar (PCFG) learning from annotated corpora, N-gram based word prediction from unannotated corpora

%inference (Bayesian or Frequentist), significance tests and logistic regression with R\\
%  Machine learning (kNN with TiMBL, Bayesian or other
%  classification/clustering with WEKA)
%\end{itemize}

\end{itemize}
\vspace{-5mm}

\subsection*{Professional membership}

\begin{itemize}
\item Qualified Member, Institute of Translation and
  Interpreting, U.K. (English to Japanese, since 2002)\\
Member, Association of Computational Linguistics (ACL)
\\
 Member, Linguistics Association of Great Britain (LAGB)
\end{itemize}

\vspace{-4mm}

\section*{Academic profile}

\vspace{-1mm}

\subsection*{Key interests}

\begin{itemize}
\item Computational modelling of language acquisition
  (phonology, semantics and syntax)\\ Natural
  language parsing\\ Syntax, semantics and their interface (HPSG,
  Dynamic Syntax, Montague Grammar)\\Information structure of Japanese

\end{itemize}

\vspace{-4mm}

\subsection*{Teaching posts}

\begin{itemize}

\item Lecturer/tutor, Imperial College London,  2003--to date\\
   M.Sc. in Scientific, Technical and Medical Translation \\
   Language Technology Module and English-Japanese Practical Translation Module

\item Teaching Assistant, King's College London. 2005--2006.\\
    B.Sc. module `Natural Language Processing in Prolog' by Professor S. Lappin.\\
   Tutorials, lectures and exam setting/marking

\vspace{-4mm}

%\item Tutor, Bath University, 2006--2008, U.K.\\
%   M.A. in Translation Studies, \\Supervision on dissertations.


\end{itemize}
%%===========================================================================%%

\vspace{-3mm}
\subsection*{Selected publications}

\papersshort{}

\vspace{-4mm}

%%---------------------------------------------------------------------------%%

%\section*{Interests}
%Cricket, Cooking, Classical music




\end{itemize}




% Footer
\begin{center}
  \begin{footnotesize}
    Last updated: \today
%\\A more updated version may be available at: \\
 %   \href{\footerlink}{\texttt{\footerlink}}
  \end{footnotesize}
\end{center}

\end{document}

%%% Local Variables:
%%% mode: latex
%%% TeX-master: t
%%% End:
