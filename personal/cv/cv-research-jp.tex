% LaTeX Curriculum Vitae Template
%
% Copyright (C) 2004-2009 Jason Blevins <jrblevin@sdf.lonestar.org>
% http://jblevins.org/projects/cv-template/
%
% You may use use this document as a template to create your own CV
% and you may redistribute the source code freely. No attribution is
% required in any resulting documents. I do ask that you please leave
% this notice and the above URL in the source code if you choose to
% redistribute this file.

\documentclass[a4paper]{article}

%\usepackage{myJapanese_xetex}

\usepackage{cv}
\usepackage{relsize}
\usepackage{fontspec}
\setmainfont[Scale=MatchLowercase]{IPA P明朝} % \rmfamily のフォント
\setsansfont[Scale=MatchLowercase]{IPAゴシック}            % \sffamily のフォント
\setmonofont[Scale=MatchLowercase]{Liberation Mono}    % \ttfamily のフォント

\def\name{佐藤 陽 {\smaller(さとう よう)}}
\def\version{research}
\def\website{{\tt\small http://homepages.stca.herts.ac.uk/\textasciitilde{}ys08aah}\\{&\tt\small http://www.satoama.co.uk/personal}}



%%% Local Variables: 
%%% mode: latex
%%% TeX-master: t
%%% End: 
\def\personal{%
\section*{Personal}

\begin{itemize}
\item Permanent resident in the United Kingdom. Japanese national.
\end{itemize}}

\def\education{%
\section*{Education}

\begin{itemize}
  \item Ph.D. Computational Linguistics, King's College London, U.K., 2008.

  \item M.Sc. Computational Linguistics (with Distinction), King's College London, U.K., 2003.

  \item Diploma in Computing, Open University, U.K., 2000.

  \item M.A. Applied Linguistics, Birkbeck College London, U.K., 1998.


  \item B.A. Philosophy, Waseda University, Japan, 1990.
\end{itemize}}


\def\languages{\section*{Language skills}
\begin{itemize}
\item Japanese (native): Lived in Japan for 25 years from birth. The
  language of daily use at home.

\vspace{-1mm}

\item  English (fluent): Have been living in London for 20 years. The language of
  daily use at work. Also the language in which I completed my higher
  degrees and conduct academic research.

\vspace{-1mm}

\item  French (advanced): Language of daily use for 4 years while living in France. Certified B2.

\vspace{-1mm}

\item  Korean (intermediate): Have been learning for five years. Can read documents
  and conduct basic conversations.
\end{itemize}

}



\def\computer{\section*{Computer skills}

Programming languages: Python, Prolog, Perl, C$^{++}$\\
OS: Unix (Linux) system operations and shell scripting\\
Text processing: sed, awk, Perl and Python\\
Other tools: \LaTeX, Git, NLTK, Weka, RNNLM

\def\researchWeb{{\tt\small http://homepages.stca.herts.ac.uk/\textasciitilde{}ys08aah}}
\def\papersWeb{\researchWeb{}{\tt\small /papers.html}}



%\def\publications_j{\section*{主要論文}

%\noindent{\smaller (全論文リストは\papersWeb{}を参照ください。またPDFのダウンロードが可能です。)}
%\bigskip
%{\fontspec{DejaVu Serif}
%\papers
%}
%}



\begin{document}

{\huge 履歴書}

\bigskip
% Place name at left
{\Large \name}

\smallskip

%{\fontspec{DejaVu Serif}Ph.D. and M.Sc. in Computational Linguistics, M.A. in Applied Linguistics}
\vspace{0.25in}

\begin{minipage}{0.4\linewidth}
住所\\
{\fontspec{DejaVu Serif}
 21 Coombe Gardens\\
  New Malden, United Kingdom\\
KT3 4AB\\

\end{minipage}
\begin{minipage}{0.43\linewidth}
  \begin{tabular}{ll}
    電話・ファックス: & {\fontspec{DejaVu Serif}(+44) (0)20-82881419} \\
    電子メール: & {\tt satoama@gmail.com} \\
    ウェブサイト: & \website{} 

  \end{tabular}
\end{minipage}

}
%\personal

\section*{主な研究テーマ及び研究関連スキル}

\begin{itemize}
\item 言語獲得・習得モデルの構築(音韻・意味・文法)\\ 文法学習アルゴリズム\\ 自然言語パーザ構築\\ 制約文法による統語・意味論及びそのインターフェイス (HPSG,
  Dynamic Syntax, Montague Grammar)

\medskip

 テキスト処理 (Python、awk及びsed)\\
  音声処理: HTKによる音響モデルの構築・音素アラインメントなど、MbrolaによるTTS\\
%  音声分析: IPA、Arpabet等に基づくトランスクリプト作成、Praat等によるスペクトログラム・プロソディ分析\\
  統計: Rを用いての推論、有意テスト及びロジスティック回帰\\
  機械学習: TiMBLを用いてのk近傍法


\end{itemize}

\section*{学歴}

\begin{itemize}

  \item 1990: 早稲田大学第一文学部哲学科人文専修卒業

  \item 1998: Birkbeck College London (英国ロンドン大学) 応用言語学修士課程修了

  \item 2000: Open University (英国放送大学) 情報科学ディプロマ修了

  \item 2003: King's College London (英国ロンドン大学) 修士課程修了(計算言語学・自然言語処理)

  \item 2008: King's College London (英国ロンドン大学) 博士課程修了(計算言語学・自然言語処理)

\end{itemize}


\section*{職歴}

\begin{itemize}

\item 1996--2002: BBC World 日本語部 

職務: BBCニュースの同時通訳及びドキュメンタリー翻訳

\item 2008--2009: 英国ロンドン大学クイーンメアリカレッジ情報科学科助手 (Research Assistant, Computer Science Dept, Queen  Mary University of London).

職務: 英国Economic and Social Science Research Council出資プロジェクト`DynDial'における対話内省略文の意味研究と パーザの実装。文法モデルにはDynamic Syntaxを採用。

\item 2009--2012: 英国ハートフォードシャー大学情報科学科助手 (Research Fellow in Computational Linguistics, University of Hertfordshire)

職務: EUプロジェクト`ITALK' (Integration and Transfer of Action and Language Knowledge in Robots)における 言語獲得モデルの構築。 主に文法習得・音韻単語獲得アルゴリズムの研究と実装   \\

\item 2012--現在: 株式会社 Vision Objects(フランス)言語処理エンジニア

職務: 手書き認識ソフトウェアの研究開発、特に形態素解析による認識精度の向上

\end{itemize}

\section*{奨学金}

\begin{itemize}

\item 2005-2008: 英国 Engineering and Physical Sciences Research Council (工学物理科学研究評議会)、 Doctoral Training Fund


\end{itemize}

\section*{教務}
\begin{itemize}

\item 2003--現在: Imperial College London 講師\\
   M.Sc. in Scientific, Technical and Medical Translation \\
   翻訳関連技術の講義と翻訳指導を担当

\item 2005--2006: King's College London. 臨時講師\\
    B.Sc. module `Natural Language Processing in Prolog' by Professor S. Lappin.\\
   講義の一部、補講および試験作成と採点を担当

\item Tutor, Bath University, 2006--2008, U.K.\\
   M.A. in Translation Studies, \\卒論指導。


\end{itemize}
%%===========================================================================%%


%\publications_j

\section*{学会運営}
\begin{itemize}
\item 2008: {\em Semantics and Pragmatics of Dialogue (SemDial) 2008, King's College London} 学会紀要編纂責任者 
\item  2011: {\em Special session on phonological word learning, International
    Conference on Development and Learning (ICDL) 2011, Frankfurt,
    Germany} 主宰

\end{itemize}

\section*{所属学会}

Association of Computational Linguistics (ACL)

Linguistics Association of Great Britain (LAGB)

自然言語処理学会

\section*{語学}

日本語 (母語)、 英語 (ネイティブ相当)、ドイツ語 (上級)、韓国語 (中級)


\section*{コンピュータ関連スキル}

言語: Python, Prolog, C$^{++}$, Haskell\\
OS: Unix (Linux) システム管理及びシェルスクリプト\\
Web: HTML5、Javascript、Python CGI、Apacheサーバ運営\\
Other tools: \LaTeX, Subversion


%%---------------------------------------------------------------------------%%

\section*{趣味}
クリケット、料理、クラシック音楽


% Footer
\begin{center}
  \begin{footnotesize}
    最終更新日時: \today%最新版は下記からダウンロードできます。
    %\href{\footerlink}{\texttt{\website{}/cv-research.pdf}}
  \end{footnotesize}
\end{center}

\end{document}
